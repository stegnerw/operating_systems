\documentclass[a4paper]{article}
\usepackage[utf8]{inputenc}
\usepackage[english]{babel}
\usepackage[]{parskip}
\usepackage{graphicx}
\usepackage{paralist} % compactitem
\usepackage{multirow}
\usepackage{longtable}
\usepackage{float}

% Custom commands
\newcommand{\figRef}[1]{Figure \ref{#1}}
\newcommand{\tabRef}[1]{Table \ref{#1}}
\newcommand{\eqRef}[1]{(\ref{#1})}

\title{EECE7095 - Homework 4}
\author{Wayne Stegner}
\date{\today}

\begin{document}
	\maketitle
	\section{Question 1}
	\par $5k jobs/month * \$2 = \$10k/month$
	\par With deadlock:
	$2 deadlocks/month * 10/2 jobs/deadlock * \$2 = \$20 loss/month$
	\par Deadlock avoidance:
	$\$10k/month * 0.1 overhead = \$1k overhead/month$
	\par \textbf{Part a.}
	\par Using the deadlock avoidance program could ensure that deadlock never
	happens, and all 5k jobs/month will still be able to run.
	This also removes some manual work, which reduces the possibility of
	human-error.
	\par \textbf{Part b.}
	\par In terms of CPU time, deadlocks only cost \$10 each time (10 jobs half
	way done @ \$2 each full job), costing only \$20/month.
	Assuming the program cost scales linearly with execution time, the deadlock
	avoidance program will add 10\% to the monthly bill, which will cost
	\$1k/month.
	The deadlock avoidance costs many times more than just restarting processes
	when a rare deadlock occurs.
	Additionally, the 20\% increase in turnaround time will probably upset
	whatever clients are waiting for the jobs, which may cause the clients to use
	another service with better turnaround time.

	\section{Question 2}
	\par \textbf{Part a.}
	\par Deadlock cannot occur in this system because it utilized pre-emption.
	If a process is using resources, it will either complete and free the
	resources, or it will block and then other processes may pre-empt its current
	resource allocation if needed.
	\par \textbf{Part b.}
	\par An indefinite block can occur if a process is continuously pre-empted.
	In the given example, P2 takes the first resource from P0, causing P0 to be
	blocked by both the first and third resources of the system.
	If processes continuously pre-empt P0 in this way, it cannot finish and will
	block forever.

	\section{Question 3}
	\begin{center}
		\begin{tabular}{| c || cccc || cccc || cccc |}
			\hline
			\multirow{2}{*}{\textbf{Thread}} &
			\multicolumn{4}{|c||}{\textbf{Allocation}} &
			\multicolumn{4}{c||}{\textbf{Max}} &
			\multicolumn{4}{c|}{\textbf{Need}} \\
			 & A & B & C & D & A & B & C & D & A & B & C & D \\
			\hline \hline
			T0     & 3 & 0 & 1 & 4 & 5 & 1 & 1 & 7 & 2 & 1 & 0 & 3 \\
			T1     & 2 & 2 & 1 & 0 & 3 & 2 & 1 & 1 & 1 & 0 & 0 & 1 \\
			T2     & 3 & 1 & 2 & 1 & 3 & 3 & 2 & 1 & 0 & 2 & 0 & 0 \\
			T3     & 0 & 5 & 1 & 0 & 4 & 6 & 1 & 2 & 4 & 1 & 0 & 2 \\
			T4     & 4 & 2 & 1 & 2 & 6 & 3 & 2 & 5 & 2 & 1 & 1 & 3 \\
			\hline
		\end{tabular}
	\end{center}
	\par \textbf{Part a.}
	\begin{center}
		\begin{tabular}{| cccc || c || l |}
			\hline
			\multicolumn{4}{|c||}{\textbf{Available}} &
			\multirow{2}{4.1em}{\textbf{Possible Threads}} &
			\multirow{2}{6em}{\textbf{Remaining Threads}} \\
			A & B & C & D &  &  \\
			\hline \hline
			0 & 3 & 0 & 1 & T2 & T0, T1, T3, T4 \\
			\hline
			3 & 4 & 2 & 2 & T1 & T0, T3, T4 \\
			\hline
			5 & 6 & 3 & 2 & T3 & T0, T4 \\
			\hline
			5 & 11 & 4 & 2 & None & T0, T4 \\
			\hline
		\end{tabular}
	\end{center}
	\par The system is in deadlock because neither T0 nor T4 can run.
	They are both waiting for more of resource D to become available.
	\par \textbf{Part b.}
	\begin{center}
		\begin{tabular}{| cccc || c || l |}
			\hline
			\multicolumn{4}{|c||}{\textbf{Available}} &
			\multirow{2}{4.1em}{\textbf{Possible Threads}} &
			\multirow{2}{6em}{\textbf{Remaining Threads}} \\
			A & B & C & D &  &  \\
			\hline \hline
			1 & 0 & 0 & 2 & T1 & T0, T2, T3, T4 \\
			\hline
			3 & 2 & 1 & 2 & T2 & T0, T3, T4 \\
			\hline
			6 & 3 & 3 & 3 & All & None \\
			\hline
		\end{tabular}
	\end{center}
	\par After running T1 and T2, enough resources become available to finish
	any of the tasks.
	Because completing a task will only allocate more resources and not take away
	resources, T0, T3, and T4 can be run in any order with no deadlock.

	\section{Question 4}
	\par Allowing two page table entires to point to the same page frame allows
	multiple processes to share the same data while saving space.
	This allows large amounts of memory to be ``copied'' without actually copying
	the whole memory page, and instead making a new page table entry point to the
	old page.
	If a shared page must be modified, the Copy on Write strategy can be used,
	so that the page will actually be copied if any of the page table entries
	must update some byte of the shared page frame.

	\section{Question 5}
	\par \textbf{Part a.}
	\par Logical memory size is $number\_of\_pages * size\_of\_each\_page$.
	With 64 pages, each containing 1024 bytes, the logical memory size is
	$64 * 1024 = 2^{6} * 2^{10} = 2^{16} bytes$.
	Thus, the logical address must contain 16 bits.
	\par \textbf{Part b.}
	\par Physical memory size is $number\_of\_frames * size\_of\_each\_frame$.
	With 32 frames, each also containing 1024 bytes, the physical memory size
	will be half of the logical memory size, or $2^{15} bytes$.
	Thus, the physical address must contain 15 bits.

	\section{Question 6}
	\par \textbf{First fit:}
	\par Each process is placed in the first partition large enough for it:
	\begin{center}
		\begin{tabular}{|l|c|c|c|c|c|c|}
			\hline
			\textbf{Partition} & 300KB & 600KB & 350KB & 750KB & 125KB \\
			\hline
			\textbf{Process}   & 115KB & 500KB & 200KB & 358KB & (empty) \\
			\hline
			\textbf{Waste}     & 185KB & 100KB & 150KB & 392KB & 125KB \\
			\hline
		\end{tabular}
	\end{center}
	\par Unable to allocate: 375KB.
	\par Total waste: 952KB.
	\par \textbf{Best fit:}
	\par Each process is placed in the available memory partition which wastes
	the least amount of space:
	\begin{center}
		\begin{tabular}{|l|c|c|c|c|c|c|}
			\hline
			\textbf{Partition} & 300KB & 600KB & 350KB & 750KB & 125KB \\
			\hline
			\textbf{Process}   & 200KB & 500KB & (empty) & 358KB & 115KB \\
			\hline
			\textbf{Waste}     & 100KB & 100KB & 350KB & 392KB & 10KB \\
			\hline
		\end{tabular}
	\end{center}
	\par Unable to allocate: 375KB.
	\par Total waste: 952KB.
	\par \textbf{Worst fit:}
	\par Each process is placed in the available memory partition which wastes
	the most amount of space:
	\begin{center}
		\begin{tabular}{|l|c|c|c|c|c|c|}
			\hline
			\textbf{Partition} & 300KB & 600KB & 350KB & 750KB & 125KB \\
			\hline
			\textbf{Process}   & (none) & 500KB & 200KB & 115KB & (none) \\
			\hline
			\textbf{Waste}     & 300KB & 100KB & 150KB & 635KB & 125KB \\
			\hline
		\end{tabular}
	\end{center}
	\par Unable to allocate: 358KB and 375KB.
	\par Total waste: 1310KB.
	\par First fit and Best fit tie for wasting the least amount of space because
	they are both unable to place the same process.
	Worst fit is the least efficient, as it wastes the most space.

	\section{Question 7}
	\par Page number is obtained by $floor(address / page\_size)$.
	Page offset is found as $address \% page\_size$.
	\par \textbf{Part a.}
	\par $Page\_number = floor(3085 / 1024) = 3$
	\par $Page\_offset = 3085 \% 1024 = 13$
	\par \textbf{Part b.}
	\par $Page\_number = floor(42095 / 1024) = 41$
	\par $Page\_offset = 42095 \% 1024 = 111$
	\par \textbf{Part c.}
	\par $Page\_number = floor(215201 / 1024) = 210$
	\par $Page\_offset = 215201 \% 1024 = 161$
	\par \textbf{Part d.}
	\par $Page\_number = floor(650000 / 1024) = 634$
	\par $Page\_offset = 650000 \% 1024 = 784$
	\par \textbf{Part e.}
	\par $Page\_number = floor(200000 / 1024) = 195$
	\par $Page\_offset = 200000 \% 1024 = 320$

	\section{Question 8}
	\par \textbf{Part a.}
	\par $Virtual\_memory\_size = number\_of\_pages * page\_size$
	\par $Number\_of\_pages = 2^{21} / 2^{11} = 2^{10}$
	\par There are 1024 entries in the conventional page table.
	\par \textbf{Part b.}
	\par $Number\_inverted\_pages = number\_physical\_frames$
	\par $Number\_inverted\_pages = physical\_memory / frame\_size =
	2^{16} / 2^{11} = 2^{5} = 32 entries$
	\par There are 32 entries in the inverted page table.
	\par \textbf{Part c.}
	\par $Maximum\_physical\_memory = 2^{16} bytes = 65536 bytes = 64KB$

	\section{Question 9}
	\par \textbf{Part a.}
	\par $Virtual\_address\_bits = log_{2}(4KB * 256) = log_{2}(2^{12} * 2^{8}) =
	20 bits$
	\par It takes 20 bits to address the logical memory space.
	\par \textbf{Part b.}
	\par $Physical\_memory\_bits = log_{2}(4KB * 64) = log_{2}(2^{12} * 2^{6}) =
	18 bits$
	\par It takes 18 bits to address the physical memory space.

	\section{Question 10}
	\par \textbf{Part a.}
	\par $Virtual\_memory\_size = number\_of\_pages * page\_size$
	\par $Number\_of\_pages = 2^{32} / 2^{12} = 2^{20} = 1048576 entries$
	\par There are 1048576 entries in the conventional page table.
	\par \textbf{Part b.}
	\par $Number\_inverted\_pages = number\_physical\_frames$
	\par $Number\_inverted\_pages = physical\_memory / frame\_size$ $=
	2^{29} / 2^{12}$
	\par $= 2^{17} entries = 131072 entries$
	\par There are 131072 entries in the inverted page table.

\end{document}

